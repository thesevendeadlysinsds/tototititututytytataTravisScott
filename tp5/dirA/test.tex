\documentclass{article}
\usepackage[utf8]{inputenc}
\usepackage[french]{babel}
\usepackage[T1]{fontenc}
\usepackage{lmodern}
\usepackage{geometry}

%\usepackage{times}

\newcommand{\texttiny}[1]{{\tiny #1}}
\newcommand{\textscriptsize}[1]{{\scriptsize #1}}
\newcommand{\textfootnotesize}[1]{{\footnotesize #1}}
\newcommand{\textsmall}[1]{{\small #1}}
\newcommand{\textnormalsize}[1]{{\normalsize #1}}
\newcommand{\textlarge}[1]{{\large #1}}
\newcommand{\textLarge}[1]{{\Large #1}}
\newcommand{\textLARGE}[1]{{\LARGE #1}}
\newcommand{\texthuge}[1]{{\huge #1}}
\newcommand{\textHuge}[1]{{\Huge #1}}

\begin{document}
%\fontsize{1pt}{0.5em}\selectfont Ce texte est plus petit que tiny
%\fontsize{40pt}{1em}{Ce texte est plus grand que huge}

\section{Exercice 3:}
\begin{enumerate}

\item
Je suis en police par défaut. 
\textsf{Je suis en police sans empattement.}
\texttt{Je suis en police à chasse fixe.}
\textrm{Je suis en police roman, explicitement.}

\item
\begin{enumerate}
\item Je ne change pas la police par défaut\ldots
\item \ldots qu'à partir de \textsf{maintenant, pour une police sans empattement \ldots}
\item \textsf{\ldots qui continue un petit moment \ldots}
\item \textsf{\ldots avant d’être finalement changée \ldots}
\item \ldots pour la police par défaut \ldots
\item \ldots puis pour \texttt{la police à chasse fixe \ldots}
\item \texttt{\ldots qui continue \ldots}
\item \texttt{\ldots encore et encore, mais pas jusqu’au bout.}
Après la liste, je retrouve la police par défaut.
\end{enumerate}
\end{enumerate}
\section{Exercice 4:}
\begin{enumerate}
\item {\Huge u}{\LARGE n}{\large i}{\normalsize e}{\small r}{\footnotesize s}{\scriptsize i}{\tiny t}{\tiny é}
\quad
{\Huge C}{\LARGE l}{\Large e}{\large r}{\normalsize m}{\small o}{\footnotesize n}{\scriptsize t}
{-}
{\Huge A}{\LARGE u}{\Large v}{\large e}{\normalsize r}{\small g}{\footnotesize n}{\scriptsize e} \quad {\small !}

\item {\Huge u}{\LARGE n}{\large i}{\normalsize e}{\small r}{\footnotesize s}{\scriptsize i}{\tiny t}{\tiny é}
\quad
{\Huge C}{\LARGE l}{\Large e}{\large r}{\normalsize m}{\small o}{\footnotesize n}{\scriptsize t}
{-}
{\Huge A}{\LARGE u}{\Large v}{\large e}{\normalsize r}{\small g}{\footnotesize n}{\scriptsize e} \quad {\small !}

Ceci est un texte normal, sans changement de taille.

\item 
\texttiny{tiny} 
\textscriptsize{scriptsize} 
\textfootnotesize{footnotesize} 
\textsmall{small} 
\textnormalsize{normalsize} 
\textlarge{large} 
\textLarge{Large} 
\textLARGE{LARGE} 
\texthuge{huge} 
\textHuge{Huge} – voilà mes différentes tailles.

\item

\end{enumerate}
\end{document}
